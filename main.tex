\documentclass{article}
\usepackage[utf8]{inputenc}

\title{Laboration – ELA405}
\author{Staffan Brickman \\ Magnus Sörenssen}
\date{February 2019}

\begin{document}

\maketitle

\section{Del 1}
\subsection{Vilken typ av signal har ni?}
För att spela in signaler använde vi MATLAB Mobile (Version 7.4.1) för iOS. Programvaran kördes på en iPhone 7S för att läsa av telefonens IMU-sensor. Vi använde en samplings frekvens på 100Hz och samlade in 1000 samples (10 sekunder) per mäting. Sammanlagt gjordes drygt 20 mätningar.
\subsection{Vilka klasser har ni?}
De uppmätta signalerna klassificerade vi i två olika klasser. Först mätte vi när vi gick vi med telefonen, sedan när vi sprang med den. Vi har alltså delat datan vi samlade in i två klasser, gående och springande.
\subsection{Varför är det viktigt att spela in flera repetitioner för varje klass?}
Anledningen till att vi gjorde så många mätningar är att vi inte vill tolka en avvikande enskild mätning som representativ för klassen. Det kunde ha inträffat om vi bara gjort ett fåtal mätningar. På så sätt kan vi jämföra signalerna inom en klass och hitta viktiga parametrar i som kan sägas representera klassen.
\newpage
\section{Del 2}
\subsection{Presentera visuellt de tider/instanser som bäst särskiljer signalerna för de två klasserna med hänsyn till medelvärde och standardavvikelse för varje klass.}
\subsection{Presentera visuellt de frekvenser som bäst särskiljer signalerna för de två klasserna med hänsyn till medelvärde och standardavvikelse för varje klass.}
\subsection{Vad kan ni dra för slutsatser från figurerna i uppgift 4 och 5?}

\end{document}
